\documentclass[12pt,a4paper]{article}
\usepackage[utf8x,utf8]{inputenc}
\usepackage[toc,page]{appendix}
%\usepackage[swedish]{babel}
\usepackage{microtype}
\usepackage{amsmath}
\usepackage{ae}
\usepackage{units}
\usepackage{url}
\usepackage{icomma}
\usepackage{listings}
\usepackage{color}
\usepackage{gauss}
\usepackage{bbm}
\usepackage{xspace}
\usepackage{float}
\usepackage{csquotes}
\newcommand{\N}{\ensuremath{\mathbbm{N}}}
\newcommand{\Z}{\ensuremath{\mathbbm{Z}}}
\newcommand{\Q}{\ensuremath{\mathbbm{Q}}}
\newcommand{\R}{\ensuremath{\mathbbm{R}}}
\newcommand{\C}{\ensuremath{\mathbbm{C}}}
\newcommand{\rd}{\ensuremath{\mathrm{d}}}
\newcommand{\id}{\ensuremath{\,\rd}}
\newcommand{\gsl}{\textsc{Gsl}\xspace}
\newcommand{\ordo}[1]{{\cal O}\left( #1 \right)}
% Want flexible vectors!
\newcommand{\myvec}[1]{\ensuremath{\mathbf{#1}}}
\usepackage{amssymb}
\usepackage{pdfpages}
\usepackage{graphicx}
\usepackage{epstopdf}
% Want beautiful tables!
\usepackage{caption}
\usepackage{dcolumn}
\setlength{\belowcaptionskip}{\baselineskip}
\newcolumntype{d}[0]{D{.}{.}{-1} }

% Special front page for Numerical Methods in Physics...
\usepackage{umuphys}            % Local umu-physics firstpage style
\author{Torbjørn Ludvigsen}
\title{Fast Fourier transforms}
\course{Numerical Methods in Physics}
\email{tolu0022@student.umu.se}
\bottomtext{FNM15\#d!DNUz\#\textbackslash0}

\begin{document}
\maketitle
%\includepdf[pages={1}]{framsida.pdf}

\begin{abstract}
\end{abstract}

\vspace{2cm}
\tableofcontents
\clearpage

\section{Introduction}
This lab is about extracting binary information from a noisy amplitude modulated (AM) signal.
The signal is generated uniquely for each student and is given as a column vector
in an ASCII-encoded data-file. 

In the next section, we present some brief theory behind discrete Fourier Transforms
and how they can be of help when we want to filter out noise from a signal.
We also present a brief description of the Gauss filter that we are going to use
once the signal is transformed.

Then, in \ref{sec:code}, we present how to implement the theory as
a computer program that can be compiled and executed.
Figures and tables discussing choice of Gauss filter-bandwidth and method for automated
bit pattern recognition is included in \ref{sec:results}.

\section{Theory}
We say that the signal is a superposition of two signals; the pure signal and the noise.
The pure signal consist of a handful of frequencies, with
the most prevalent (largest amplitude) of them being a so called ``carrying frequency'',
$f_c$, and two ``modulating frequencies'' or ``side bands'', $f_c\pm\Delta f/2$.
Together, they make the pure signal look like
a series of wave packets with equal length, that take on one of two
different amplitudes per period.

These two amplitudes are what sepparates the ``1''s in the signal from the ``0''s,
so finding the wave packets and their amplitude is the key to finding the
information hidden in the signal.

The noise is simple white noise, a signal with no particular periodicity but lots of frequencies.
The noise is as strong as the pure signal, but randomly distributed, so the wave packets
are impossible to see with the naked eye. We want to remove the noise from the signal
with a filter.

\subsection{Discrete Fourier Transform}
The Fourier Theorem states that every periodic signal can be decomposed into
a series of sines and cosines with different frequencies and coefficients (amplitudes).
Since the distribution of these coefficients (the frequency distribution)
is dramatically different in our pure signal and in the noise,
we would like to do decompose our raw signal into such a series so that we can
investigate and change the coefficients.

A transformation that extracts the coefficients from a signal, $h(t)$ ``amplitude per time'',
is called a Fourier Transform. Since our signal is discrete, we use a Discrete Fourier
Transform:
\begin{equation}\label{eq:ft}
  H_n = \sum_{k=0}^{N-1} h_ke^{2\pi ikn/N},
\end{equation}
where $h_k$ is our discrete signal in time, $N$ is the number of samples we have and
$H_n$ is the amplitude of the $n'th$ frequency in the signal.

The problem of removing the noise can now be expressed as
``finding the $f_c$ and the side bands and making their coefficients
dominate all the others''. This operation is called filtering, and is
described in section \ref{sec:gauss}.

When filtering is done,
we still need to reconstruct the time-dependent pure signal, $p_k$,
from our new set of
coefficients, $P_n$, in order to see the wave packets.
This reconstruction is called
an Inverse Fourier Transform, and its discrete version can be defined like:
\begin{equation}\label{eq:ift}
  p_k = \frac1N\sum_{i=0}^{N-1}P_ne^{-2\pi ikn/N}.
\end{equation}


\subsection{The Gaussian Filter}\label{sec:gauss}
This section will be about getting from $H_n$ in \eqref{eq:ft}
to approximately $p_k$ in \eqref{eq:ift}.
% HERE....
In our case, we want to multiply $G$ and $H$ before inverse transforming,
giving the expression
\begin{align}
  (g*h)_k &= \frac1N\sum_{i=0}^{N-1}G_n(\sum_{k=0}^{N-1} h_ke^{2\pi ikn/N})e^{-2\pi ikn/N}\\
          &= \frac1N\sum_{i=0}^{N-1}(GH)_ne^{-2\pi ikn/N},
\end{align}
where $(g*h)_k$ is the frequency-filtered version of signal $h_k$ and $g$
is the inverse Fourier Transform of $H$.

This process is called ``filtering'', and we denote the
filtered coefficients like $(GH)_n$ here.%\footnote{Note that $*$ means convolution, not multiplication. More on that in Section \ref{sec:gauss}.}
That is, we simply multiply all coefficients $H_n$ with
$G$, a function of frequency that is low over the
frequencies we want to filter out.

When $G_n$ is a Gauss curve, this whole operation on $h_k$ is called ``Gaussian filtering''.
By centering the Gauss curve above the 

It is not strictly needed for 



\section{Code}\label{sec:code}
% describe software, abstract point of view
% describe external functions used

\subsection{Numerical methods}\label{sec:nm}
% give short overview of Cooley-Turk FFT algorithm here
% link equations in Theory to the written code

\subsection{Program}
% how to compile and use program

\section{Results}\label{sec:results}
% figures, tables, explanatory text
% answer questions posed in introduction
% make sure all issues mentioned in spec are discussed


\begin{figure}
  \begin{small}
  \centering
  \input{different_filters.tex}
  \caption{
    Visualizations of Gaussian filters with different bandwidths
    along with the the resulting signals after using them (expressed
    in absolute values).
    The frequencies are on the $x$-axis and the top left plot shows the
    the \emph{unfiltered} version of what is filtered in the three other plots.
    Notice that the highest frequencies are cut out of this plot since
    all filters zero them out anyways.
    Also note that negative frequencies not shown in this plot, they would
    look mirrored across the $y$-axis if they were included.
    We see that a bandwidth of $\unit[64]{Hz}$ lets almost only the carrying frequency
    through, and damps the modular frequencies down to amplitudes of the white
    noise. The filter with $\Delta f = \unit[128]{Hz}$ supresses a lot of noise,
    but also damps out ca $\unit[50]{\%}$ of the modular frequencies 
    (those at $\unit[896]{Hz}$ and $\unit[1152]{Hz}$).
    The filter with
    $\Delta f = \unit[256]{Hz}$ also supresses most of the noise-frequencies
    and damps the modular frequencies with only about $\unit[15]{\%}$.
    Plots of the signals after they've been filtered and transformed back
    to the time domain is found in Figure \ref{fig:different_bandwidths}.
    Since the filter is implemented programmatically, it would have been easy to
    write a filter that simply picked the information carrying frequencies.
  }
  \label{fig:different_filters}
  \end{small}
\end{figure}

\begin{figure}
  \begin{small}
  \centering
  \input{different_bandwidths.tex}
  \caption{
    The first byte of the AM-modulated binary data after transforming,
    filtering and inverse transforming. $\Delta f$ denotes the bandwidth
    the Gauss filter that has been used (see Figure \ref{fig:different_filters}).
    We see that a filter with a too narrow bandwidth, in this case $\Delta f = \unit[64]{Hz}$,
    makes overall amplitude low, and bytes look blurred/smeared out.
    A too wide bandwidth, in this case $\Delta f = \unit[512]{Hz}$ ,allows
    the noise to disturb the signal making it look somewhat irregular.
  }
  \label{fig:different_bandwidths}
  \end{small}
\end{figure}

\begin{figure}
  \begin{small}
  \centering
  \input{filtered_with_bandwidth_256.tex}
  \caption{The first tenth of the signal, filtered with a Gaussian
    filter using bandwidth
    $\Delta f = \unit[256]{Hz}$ and $f_c = \unit[1024]{Hz}$.
    Notice that time units are assumed to be such that the carrier wave
    get frequency $\unit[1024]{Hz}$ and the modulation frequency
    is $\unit[8/1024]{Hz}$.
    We se clearly the two types of
    wave packets that decode the binary information.
    The wave packets with amplitudes
    around $0.8$ can be interpreted as ``0''s
    and the ones with amplitudes around $2.4$ can be interpreted as ``1''s.
    The first eight bits in this pattern is then the binary number $01000110$.
    In decimal this is $70$ and interpreted as a
    ASCII-decoded letter, this is an \textit{F}.
  }
  \label{fig:filtered_with_bandwidth_256}
  \end{small}
\end{figure}



\section{Conclusions}
% validity of results?
% possible implications?
% possible improvements?


\clearpage
\begin{thebibliography}{99}

%\bibitem{sampl}
%  W. H. Press, S. A. Teukolsky, W. T. Vetterling, and
%  B. P. Flannery, \emph{Numerical Recipes in C}, 2nd ed. (Cambridge
%  University Press, Cambridge, 1992).

\end{thebibliography}

\appendix
\section{Appendices}

\end{document}

%\begin{figure}
%  \begin{small}
%  \centering
%  \input{task<nr>.tex}
%  \caption{
%  }
%  \label{fig:task<nr>}
%  \end{small}
%\end{figure}

%\begin{table}
%  \begin{center}
%    \caption{<caption of a table>}
%    \label{table:task<nr>}
%    \begin{tabular}{ c | d }
%      Option      & value      \\
%      \hline
%      <option1>   & 1.2  \\
%    \end{tabular}
%  \end{center}
%\end{table}

%\begin{thebibliography}{99}
%\bibitem{octave_wiki}
%  Octave Wiki\\
%  \emph{Tips and Tricks}\\
%  \url{http://wiki.octave.org/Tips_and_tricks}\\
%  Date: 10 -- 11 -- 2014
%\end{thebibliography}
