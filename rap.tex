\documentclass[12pt,a4paper]{article}
\usepackage[utf8x,utf8]{inputenc}
\usepackage[toc,page]{appendix}
%\usepackage[swedish]{babel}
\usepackage{microtype}
\usepackage{amsmath}
\usepackage{ae}
\usepackage{units}
\usepackage{url}
\usepackage{icomma}
\usepackage{listings}
\usepackage{color}
\usepackage{gauss}
\usepackage{bbm}
\usepackage{xspace}
\usepackage{float}
\usepackage{csquotes}
\newcommand{\N}{\ensuremath{\mathbbm{N}}}
\newcommand{\Z}{\ensuremath{\mathbbm{Z}}}
\newcommand{\Q}{\ensuremath{\mathbbm{Q}}}
\newcommand{\R}{\ensuremath{\mathbbm{R}}}
\newcommand{\C}{\ensuremath{\mathbbm{C}}}
\newcommand{\rd}{\ensuremath{\mathrm{d}}}
\newcommand{\id}{\ensuremath{\,\rd}}
\newcommand{\gsl}{\textsc{Gsl}\xspace}
\newcommand{\ordo}[1]{{\cal O}\left( #1 \right)}
% Want flexible vectors!
\newcommand{\myvec}[1]{\ensuremath{\mathbf{#1}}}
\usepackage{amssymb}
\usepackage{pdfpages}
\usepackage{graphicx}
\usepackage{epstopdf}
% Want beautiful tables!
\usepackage{caption}
\usepackage{dcolumn}
\setlength{\belowcaptionskip}{\baselineskip}
\newcolumntype{d}[0]{D{.}{.}{-1} }

% Special front page for Numerical Methods in Physics...
\usepackage{umuphys}            % Local umu-physics firstpage style
\author{Torbjørn Ludvigsen}
\title{Fast Fourier transforms}
\course{Numerical Methods in Physics}
\email{tolu0022@student.umu.se}
%\bottomtext{Extracting Information from a Noisy Signal}

\begin{document}
\maketitle
%\includepdf[pages={1}]{framsida.pdf}

\begin{abstract}
\end{abstract}

\vspace{2cm}
\tableofcontents
\clearpage

\section{Introduction}
This lab is about extracting information from a noisy signal.
One can think of the signal as something that one would get from
recording an amplitude modulated (AM) radio-signal that carried binary information.
We say that the noisy signal consist of two parts, pure signal and noise.
This makes sense in all cases where a signal is represented by some measurable
physical entity The $2^{nd}$ law of thermodynamics tells us that the
probability of increased entropy (could be interpreted as disorder or ``noise'') is very high.

The pure signal has only one frequency, we call that the information-carrying
frequency $\nu_0$. Since the information is binary, the pure signal
takes on one of two different amplitudes, call them $a_{high}$ and $a_{low}$.
One period with high amplitude in the pure signal encodes a 1, one period with
low amplitude encodes a 0.

The signal is given as a discrete column vector and is generated uniquely for each student.
The column vector itself is given in as lines in an ASCII-encoded file.
In the next section, we present the theory behind how FFT-techniques can be used to decode such a noisy signal.
Then, in \ref{sec:code}, we present the tools we are going to use to implement the theory as
procedures expressed as a computer program that can be compiled and executed.
Figures and tables discussing <insert issues mentioned in spec here> is
included in \ref{sec:results}.


\section{Theory}\label{sec:code}
% give equations that program solves

\section{Code}
% describe software, abstract point of view
% describe external functions used

\subsection{Numerical methods}
% give short overview of Cooley-Turk FFT algorithm here
% link equations in Theory to the written code

\subsection{Program}
% how to compile and use program

\section{Results}\label{sec:results}
% figures, tables, explanatory text
% answer questions posed in introduction
% make sure all issues mentioned in spec are discussed

\section{Conclusions}
% validity of results?
% possible implications?
% possible improvements?


\clearpage
\begin{thebibliography}{99}

%\bibitem{sampl}
%  W. H. Press, S. A. Teukolsky, W. T. Vetterling, and
%  B. P. Flannery, \emph{Numerical Recipes in C}, 2nd ed. (Cambridge
%  University Press, Cambridge, 1992).

\end{thebibliography}

\appendix
\section{Appendices}

\end{document}

%\begin{figure}
%  \begin{small}
%  \centering
%  \input{task<nr>.tex}
%  \caption{
%  }
%  \label{fig:task<nr>}
%  \end{small}
%\end{figure}

%\begin{table}
%  \begin{center}
%    \caption{<caption of a table>}
%    \label{table:task<nr>}
%    \begin{tabular}{ c | d }
%      Option      & value      \\
%      \hline
%      <option1>   & 1.2  \\
%    \end{tabular}
%  \end{center}
%\end{table}

%\begin{thebibliography}{99}
%\bibitem{octave_wiki}
%  Octave Wiki\\
%  \emph{Tips and Tricks}\\
%  \url{http://wiki.octave.org/Tips_and_tricks}\\
%  Date: 10 -- 11 -- 2014
%\end{thebibliography}
